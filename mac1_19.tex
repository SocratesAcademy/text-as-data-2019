\documentclass{beamer}

%\usepackage[table]{xcolor}
\mode<presentation> {
  \usetheme{Boadilla}
%  \usetheme{Pittsburgh}
%\usefonttheme[2]{sans}
\renewcommand{\familydefault}{cmss}
%\usepackage{lmodern}
%\usepackage[T1]{fontenc}
%\usepackage{palatino}
%\usepackage{cmbright}
  \setbeamercovered{transparent}
\useinnertheme{rectangles}
}
%\usepackage{normalem}{ulem}
%\usepackage{colortbl, textcomp}
\setbeamercolor{normal text}{fg=black}
\setbeamercolor{structure}{fg= black}
\definecolor{trial}{cmyk}{1,0,0, 0}
\definecolor{trial2}{cmyk}{0.00,0,1, 0}
\definecolor{darkgreen}{rgb}{0,.4, 0.1}
\usepackage{array}
\beamertemplatesolidbackgroundcolor{white}  \setbeamercolor{alerted
text}{fg=red}

\setbeamertemplate{caption}[numbered]\newcounter{mylastframe}

%\usepackage{color}
\usepackage{tikz}
\usetikzlibrary{arrows}
\usepackage{colortbl}
%\usepackage[usenames, dvipsnames]{color}
%\setbeamertemplate{caption}[numbered]\newcounter{mylastframe}c
%\newcolumntype{Y}{\columncolor[cmyk]{0, 0, 1, 0}\raggedright}
%\newcolumntype{C}{\columncolor[cmyk]{1, 0, 0, 0}\raggedright}
%\newcolumntype{G}{\columncolor[rgb]{0, 1, 0}\raggedright}
%\newcolumntype{R}{\columncolor[rgb]{1, 0, 0}\raggedright}

%\begin{beamerboxesrounded}[upper=uppercol,lower=lowercol,shadow=true]{Block}
%$A = B$.
%\end{beamerboxesrounded}}
\renewcommand{\familydefault}{cmss}
%\usepackage[all]{xy}

\usepackage{tikz}
\usepackage{lipsum}

 \newenvironment{changemargin}[3]{%
 \begin{list}{}{%
 \setlength{\topsep}{0pt}%
 \setlength{\leftmargin}{#1}%
 \setlength{\rightmargin}{#2}%
 \setlength{\topmargin}{#3}%
 \setlength{\listparindent}{\parindent}%
 \setlength{\itemindent}{\parindent}%
 \setlength{\parsep}{\parskip}%
 }%
\item[]}{\end{list}}
\usetikzlibrary{arrows}
%\usepackage{palatino}
%\usepackage{eulervm}
\usecolortheme{lily}

\newtheorem{com}{Comment}
\newtheorem{lem} {Lemma}
\newtheorem{prop}{Proposition}
\newtheorem{thm}{Theorem}
\newtheorem{defn}{Definition}
\newtheorem{cor}{Corollary}
\newtheorem{obs}{Observation}
 \numberwithin{equation}{section}


\title[Machine Learning] % (optional, nur bei langen Titeln nötig)
{Machine Learning for Social Sciences}

\author{Justin Grimmer}
\institute[Stanford University]{Professor\\Department of Political Science \\  University of Chicago}
\vspace{0.3in}

\date{April 2nd, 2019}

\begin{document}
\begin{frame}
\titlepage
\end{frame}





\begin{frame}

\huge 

Machine Learning in the Social Sciences 

\begin{itemize}
\item[-] Discovery 
\item[-] Measurement
\item[-] Causal Inference
\end{itemize}  

\end{frame}


\begin{frame}

\huge

Machine learning $\leadsto$ powerful, but important to recognize limitations


\end{frame}




\begin{frame}
\frametitle{Online Advertisements}



\begin{itemize}
\item[-] Online ads: \alert{billions} of revenue
\item[-] Last click attribution: ads ``get credit" if last thing you see before you buy
\item[-] Goal: optimize probability my ad is the last one clicked
\end{itemize}

Optimized, but for the task you choose


\end{frame}  



\begin{frame}
\frametitle{Voter Targeting Decisions}

Campaigns: exert effort to mobilize voters
\begin{itemize}
\item[-] Voter lists, consumer data, and proprietary surveys to target
\item[-] Hersh 2015: limitations to voter file, depends on state
\item[-] \alert{Merge}: hard to combine data from different sources
\item[-] \alert{Clean}: hard to know if someone has moved or just not voting
\item[-] \alert{Target}: hard to run experiment during campaign to determine who to target
\end{itemize}  

You work with the data you have 





\end{frame}


 \begin{frame}
 \frametitle{Twitter and the 280 Character Experiment}

\only<1-2,3->{
 \begin{itemize}
 \item[-] {\tt Twitter.com} increase engagement with longer tweets \pause 
 \invisible<1>{\item[-] \alert{Experiment}: limited roll out, observe effect with small (1\%) treated group}\pause 
 \invisible<1-3>{\item[-] Now:\emph{no one notices} the 280 characters} \pause 
 \end{itemize}  
}

 \only<3>{\scalebox{0.5}{\includegraphics{bears1.png}}}
\pause 
 \invisible<1-4>{Experiments (and analysis) provide specific information that may not relate to actual quantity of interest} \pause 

 \invisible<1-5>{Subsequent work (not with experiment) shows effects on politeness, tone on twitter}




 \end{frame}


\begin{frame}
\frametitle{Machine Learning and ``Bias"}

Machine learning methods can mitigate bias in decision making
\begin{itemize}
\item[-] Kleinberg et al ``Human Decisions and Machine Predictions" $\leadsto$ Make better bail decisions using machine learing
\item[-] Bansak et al $\leadsto$ machine learning places refugees in better areas
\end{itemize}

Machine learning methods can inherent (and amplify) biases in decision making
\begin{itemize}
\item[-] Caliskan et al ``Semantics derived automatically from language corpora contain human-like biases" $\leadsto$ machine learning can inherent human biases
\end{itemize}  

Machine learning is not a panacea for human biases


\end{frame}



\begin{frame}
\frametitle{Text and Political Science}

A pre-2000's view of text in social science
\begin{itemize}
\item[-] Social interaction often occurs in texts \pause
\invisible<1>{\item[-] Social Scientists avoided studying texts/speech} \pause
\invisible<1-2>{\item[-] Why?} \pause
\begin{itemize}
\invisible<1-3>{\item[-] Hard to find} \pause
\invisible<1-4>{\item[-] Time Consuming} \pause
\invisible<1-5>{\item[-]  Not generalizable (each new data set...new coding scheme)} \pause
\invisible<1-6>{\item[-]  Difficult to store/search} \pause
\invisible<1-7>{\item[-] Idiosyncratic to coders/researcher} \pause
\invisible<1-8>{\item[-] Statistical methods/algorithms, computationally intensive}
\end{itemize}
\end{itemize}

\end{frame}




\begin{frame}

A post-2000's view of text in social science:
\vspace{0.25in}

\invisible<1>{ Massive collections of texts are increasingly used as a data source in social science: }


\begin{itemize}
\invisible<1-2>{\item[-] Congressional speeches, press releases, newsletters, ...}
\invisible<1-3>{\item[-]  Facebook posts, tweets, emails, cell phone records, ...}
\invisible<1-4>{\item[-]  Newspapers, magazines, news broadcasts, ...  }
\invisible<1-5>{\item[-] Foreign news sources, treaties, sermons, fatwas, ...}
\end{itemize}

\pause \pause \pause \pause \pause


\end{frame}

\begin{frame}
%
\footnotesize
Why? \pause
\begin{itemize}
\invisible<1>{\item[-] Massive increase in availability of unstructured text (10 minutes of worldwide email = 1 LOC )} \pause
\invisible<1-2>{\item[-] Cheap storage: 1956: \$10,000 megabyte.  2019: $<<<<<<<$ \$0.0001 per megabyte  (Unless you're sending an SMS)  } \pause
\invisible<1-3>{\item[-] Explosion in methods and programs to analyze texts} \pause
\begin{itemize}
\invisible<1-4>{\item[-] Generalizable:  one method can be used across many methods and to unify collections of texts} \pause
\invisible<1-5>{\item[-] Systematic: parameters/statistics demonstrate how models make coding decisions} \pause
\invisible<1-6>{\item[-] Cheap: easily applied to many new collections of texts, computing power is inexpensive} \pause
\end{itemize}
\invisible<1-7>{\item[-] \alert{Unchanged Demand}: Social life (politics, economic exchanges, social interactions) occurs in \alert{texts} } \pause
\begin{itemize}
\invisible<1-8>{\item[-] Laws} \pause
\invisible<1-9>{\item[-] Treaties } \pause
\invisible<1-10>{\item[-] News media} \pause
\invisible<1-11>{\item[-] Campaigns} \pause
\invisible<1-12>{\item[-] Political pundits} \pause
\invisible<1-13>{\item[-] Petitions } \pause
\invisible<1-14>{\item[-] Press Releases} \pause
\invisible<1-15>{\item[-] ...}
\end{itemize}
\end{itemize}
\end{frame}


\begin{frame}
\frametitle{What Can Text Methods Do?}

Haystack metaphor:\pause  \invisible<1>{\alert{ Improve Reading}} \pause
\begin{itemize}
\invisible<1-2>{\item[-]  Interpreting the meaning of a sentence or phrase $\leadsto$ Analyzing a straw of hay} \pause
\begin{itemize}
\invisible<1-3>{\item[-]Humans: amazing (Straussian political theory, analysis of English poetry)
\item[-] Computers: struggle } \pause
\end{itemize}
\invisible<1-4>{\item[-] Comparing, Organizing, and Classifying Texts$\leadsto$ Organizing hay stack} \pause
\begin{itemize}
\invisible<1-5>{\item[-] Humans: terrible.  Tiny active memories
\item[-] Computers: amazing$\leadsto$ largely what we'll discuss today} \pause
\end{itemize}
\end{itemize}
\invisible<1-6>{What automated text methods don't do:} \pause
\begin{itemize}
\invisible<1-7>{\item[-] Develop a comprehensive statistical model of language
\item[-] Replace the need to read
\item[-] Develop a single tool + evaluation for all tasks }
\end{itemize}
\end{frame}


\begin{frame}
\frametitle{Texts are Deceptively Complex}


{\tt We've got some difficult days ahead.  But it doesn't matter with me now. Because I've been to the mountaintop. And I don't mind. Like anybody, I would like to live a long life. Longevity has its place. But I'm not concerned about that now. }

\pause

\begin{itemize}
\invisible<1>{\item[-] Who is the {\tt I} ?} \pause
\invisible<1-2>{\item[-] Who is the {\tt We}? } \pause
\invisible<1-3>{\item[-] What is the {\tt mountaintop} (literal?)} \pause
\end{itemize}

\invisible<1-4>{Texts$\leadsto$ high dimensional, not self contained}


\end{frame}


\begin{frame}
\frametitle{Texts are Surprisingly Simple}

(Lamar Alexander (R-TN) Feb 10, 2005)
\begin{table}
\begin{tabular}{ll}
\hline
 Word & No. Times Used in Press Release \\
 \hline
 department &     12\\
  grant  &      9 \\
  program &   7 \\
  firefight &  7 \\
  secure  &    5 \\
  homeland  &  4 \\
  fund   &    3 \\
 award  &    2 \\
 safety  &     2 \\
 service &    2 \\
 AFGP   &    2 \\
 support &   2 \\
equip  &    2 \\
applaud  &  2\\
 assist  &   2\\
prepared &  2 \\
\hline
\end{tabular}
\end{table}



\end{frame}




\begin{frame}
\frametitle{Texts are Surprisingly Simple (?)}

{\tt US  Senators Bill Frist  (R-TN)  and Lamar Alexander (R-TN) today applauded the U S  Department of Homeland Security for awarding a \$8,190 grant to the Tracy City Volunteer Fire Department under the 2004 Assistance to Firefighters Grant Program's  (\alert{AFGP})  Fire Prevention and Safety Program...}


\end{frame}



\begin{frame}
\frametitle{Not just for ``big data" }

\pause

\invisible<1>{Manually develop categorization scheme for partitioning small (100) set of documents} \pause

  \begin{itemize}
\invisible<1-2>{ \item[-] Bell$(n) = $ number of ways of partitioning $n$ objects}
    \pause
   \invisible<1-3>{ \item[-] Bell$(2)=2$ (AB, A B)}\pause
  \invisible<1-4>{ \item[-] Bell$(3)=5$ (ABC, AB C, A BC, AC B, A B
   C)} \pause
   \invisible<1-5>{ \item[-] Bell$(5)=52$}\pause
   \invisible<1-6>{ \item[-] Bell(100)}\pause \invisible<1-7>{$\approx 4.75 \times
   10^{115}$ partitions} \pause
   \invisible<1-8>{\item[-] \alert{Big Number}:}\pause  \\
   \invisible<1-9>{7 Billion RAs} \pause  \\
   \invisible<1-10>{ Impossibly Fast (enumerate one clustering every millisecond)} \pause  \\
   \invisible<1-11>{ Working around the clock (24/7/365)}\pause  \\
        \invisible<1-12>{$\approx 1.54\times 10^{84} \times $} \invisible<1-13>{($14,000,000,000$)} \invisible<1-14>{years}
        \pause \pause\pause
    \end{itemize}

\invisible<1-15>{\alert{Machine Learning methods can help with even small problems}}

\end{frame}


\begin{frame}
\frametitle{Course Plan}

\begin{itemize}
\item[-] Preliminaries: Acquiring Text and Feature Engineering
\item[-] Discovery 
\begin{itemize}
\item[-] Regular Expressions and Vector Space Model of Text
\item[-] Unsupervised Clustering 
\item[-] Topic Models
\item[-] Embeddings 
\item[-] Fictitious Prediction Problems
\end{itemize}  
\item[-] Measurement
\begin{itemize}
  \item[-] Hand Coding
  \item[-] Dictionary Methods
  \item[-] LASSO and Ridge
  \item[-] Naive Bayes and ReadMe
  \item[-] Boosting, Bagging, and Ensembles
  \item[-] Structural Topic Models for Measurement
\end{itemize}
\item[-] Causal Inference
\begin{itemize}
\item[-] Text as Intervention
\item[-] Text as Response and as Covariate
\end{itemize}  
\end{itemize}  

\end{frame}


\begin{frame}
\frametitle{Course Evaluation Plan}

Three (Equal) Parts to Evaluation 

\begin{itemize}
  \item[1)] 5 homeworks. 
  \begin{itemize}
    \item[-] Collaborate with folks in class
    \item[-] But write up your own work
    \item[-] Goal: (1) deeper understanding of the statistical methods (2) develop programming skills and (3) learn how to apply techniques from class to your own work
  \end{itemize}
  \item[2)] Class Participation 
  \item[3)] Poster Session + Paper
\end{itemize}  


\end{frame}


\begin{frame}
\frametitle{Poster Session + Paper}

Goal: create \emph{publishable} research output\\
Work in groups (2-3 people), apply methods from the class \\
Sequence:
\begin{itemize}
\item[-] Initial project selection/question: April 16th.  
\item[-] Data set collected, ready to analyze: May 7th
\item[-] Initial analyses/Write Up: May 16th
\item[-] Final Meeting with me to discuss project:  May 28th
\item[-] \alert{Poster Session}: June 4th
\item[-] Paper due by the end of final exam period
\end{itemize} 


\alert{I want to work with you to make publishable research}



\end{frame}

\begin{frame}
\frametitle{Opportunity for Faculty Collaboration}

\textbf{Anna Grzymala-Busse}
 ``Looking for fairly simple sentiment analysis of official pronouncements and declarations (bulls) of the Catholic popes from about 8th to the 21st century. I’m specifically looking for how official church views on the state have changed over time: how the church views the claims of rulers/ monarchs, its views on different forms of government (monarchy/ subsidiarity/ democracy etc), and when and how it sees the state as a rival or a complement in areas such as administration, the naming of clerics, education, taxes, health care, poverty relief, etc. How consistent are the Church’s views? How do they change in response to threats such as the Black Plague or the Reformation? "


\end{frame}



\begin{frame}
\frametitle{Course Content}

Prerequisites:
\begin{itemize}
\item[1)] Must have: Linear Regression, Mathematical Statistics, background in {\tt R}, {\tt Python} or related language
\item[2)] (Very) Nice to have: Likelihood Theory, Causal Inference, and related courses
\end{itemize}  

Technical class:
\begin{itemize}
\item[-] Hard work: time spent on programming, problem sets, and research
\item[-] Time consuming: please set aside time to work on this class
\item[-] \alert{Everyone can succeed}
\end{itemize}  

Questions: Smartest person in the room rule



\end{frame}


\begin{frame}
\huge

Thursday: Feature Representations


\end{frame}


\end{document}

\begin{frame}

\huge
Six principles for Machine Learning and Social Science

\end{frame}


\begin{frame}
\huge 
Social Science Theories Are the Starting Point For Analyis  \\

\end{frame}

\begin{frame}

\huge 

Text as Data Methods Do Not Replace Humans, They Augment Them


\end{frame}


\begin{frame}

\huge 

Building, Refining, and Testing Social Science Theories Requires Iteration and Sequence



\end{frame}

\begin{frame}

\huge 
Text as Data Methods Distill Generalizations from Language  


\end{frame}


\begin{frame}

\huge 

Different Text as Data Tasks Require Different Methods

\end{frame}






\begin{frame}
\huge 

Task specific and Theory Specific Validation
\end{frame}



\begin{frame}

\huge 

Thursday: Feature Representations


\end{frame}



\end{document}
