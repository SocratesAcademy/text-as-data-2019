\documentclass{beamer}

%\usepackage[table]{xcolor}
\mode<presentation> {
  \usetheme{Boadilla}
%  \usetheme{Pittsburgh}
%\usefonttheme[2]{sans}
\renewcommand{\familydefault}{cmss}
%\usepackage{lmodern}
%\usepackage[T1]{fontenc}
%\usepackage{palatino}
%\usepackage{cmbright}
  \setbeamercovered{transparent}
\useinnertheme{rectangles}
}
%\usepackage{normalem}{ulem}
%\usepackage{colortbl, textcomp}
\setbeamercolor{normal text}{fg=black}
\setbeamercolor{structure}{fg= black}
\definecolor{trial}{cmyk}{1,0,0, 0}
\definecolor{trial2}{cmyk}{0.00,0,1, 0}
\definecolor{darkgreen}{rgb}{0,.4, 0.1}
\usepackage{array}
\newcommand{\argmin}{\arg\!\min}
\beamertemplatesolidbackgroundcolor{white}  \setbeamercolor{alerted
text}{fg=red}

\setbeamertemplate{caption}[numbered]\newcounter{mylastframe}

%\usepackage{color}
\usepackage{tikz}
\usetikzlibrary{arrows}
\usepackage{colortbl}
%\usepackage[usenames, dvipsnames]{color}
%\setbeamertemplate{caption}[numbered]\newcounter{mylastframe}c
%\newcolumntype{Y}{\columncolor[cmyk]{0, 0, 1, 0}\raggedright}
%\newcolumntype{C}{\columncolor[cmyk]{1, 0, 0, 0}\raggedright}
%\newcolumntype{G}{\columncolor[rgb]{0, 1, 0}\raggedright}
%\newcolumntype{R}{\columncolor[rgb]{1, 0, 0}\raggedright}

%\begin{beamerboxesrounded}[upper=uppercol,lower=lowercol,shadow=true]{Block}
%$A = B$.
%\end{beamerboxesrounded}}
\renewcommand{\familydefault}{cmss}
%\usepackage[all]{xy}

\usepackage{tikz}
\usepackage{lipsum}

 \newenvironment{changemargin}[3]{%
 \begin{list}{}{%
 \setlength{\topsep}{0pt}%
 \setlength{\leftmargin}{#1}%
 \setlength{\rightmargin}{#2}%
 \setlength{\topmargin}{#3}%
 \setlength{\listparindent}{\parindent}%
 \setlength{\itemindent}{\parindent}%
 \setlength{\parsep}{\parskip}%
 }%
\item[]}{\end{list}}
\usetikzlibrary{arrows}
%\usepackage{palatino}
%\usepackage{eulervm}
\usecolortheme{lily}
\newtheorem{com}{Comment}
\newtheorem{lem} {Lemma}
\newtheorem{prop}{Proposition}
\newtheorem{thm}{Theorem}
\newtheorem{defn}{Definition}
\newtheorem{cor}{Corollary}
\newtheorem{obs}{Observation}
 \numberwithin{equation}{section}

%\usepackage[latin1]{inputenc}
\title[Machine Learning] % (optional, nur bei langen Titeln nötig)
{Machine Learning}

\author{Justin Grimmer}
\institute[Stanford University]{Professor\\Department of Political Science \\  Stanford University}
\vspace{0.3in}


\date{April 11th, 2019}%[Big Data Workshop] 
%\date{\today}


\begin{document}
\begin{frame}
\titlepage
\end{frame}

\begin{frame}

\huge 

Last class: vector space model of text (tf-idf)\\

This class: statistical model of text \\

Goal: Discovery (some new conceptualization)




\end{frame}





\begin{frame}
\frametitle{Building Models vs Algorithms}

\Large

Today we're building statistical models \pause \\
\invisible<1>{ \alert{algorithms}$\leadsto$ models without a data generating process\\} \pause 

\invisible<1-2>{False debate.  Both are useful\\} \pause 

\invisible<1-3>{We will often use statistics, because:} \pause 

\begin{itemize}
\invisible<1-4>{\item[1)] Clarity of assumptions} \pause 
\invisible<1-5>{\item[2)] Machinery to think about optimization, uncertainty, and sensitivity$\leadsto$ automatic objective function} \pause 
\invisible<1-6>{\item[3)] Easier to extend---leverage technology to generalize the model} \pause 
\end{itemize}



\invisible<1-7>{Probability model$\leadsto$ form basis of statistical approaches} 

\end{frame}


\begin{frame}
\frametitle{Unigram Model of Language}

\Large 

Assume we have a $3$ word \alert{vocabulary}\pause \invisible<1>{$\leadsto$ $3$ words that we might speak. } \pause \\
\invisible<1-2>{\alert{Bag of Words}$\leadsto$ each word is an independent draw over $3$ words} \pause 
\begin{itemize}
\invisible<1-3>{\item[-] \alert{Improbable model of language creation}} \pause 
\invisible<1-4>{\item[-] Complex dependency structure of text} \pause 
\invisible<1-5>{\item[-] Improbable $\neq$ useless} 
\end{itemize}


\end{frame}


\begin{frame}
\frametitle{Unigram Model of Language}

Suppose we are drawing a word $\boldsymbol{X}_{i} = (X_{i1}, X_{i2}, X_{i3}) $ \pause 
\begin{eqnarray}
\invisible<1>{p(\boldsymbol{X}_{i}  =   (1, 0, 0)) &= &  \theta_{1} \nonumber}\pause  \\
\invisible<1-2>{p(\boldsymbol{X}_{i}  =   (0, 1, 0)) & =&  \theta_{2} \nonumber}\pause  \\
\invisible<1-3>{p(\boldsymbol{X}_{i} =  (0,0,1)) &= & \theta_{3} = 1 - \theta_{2} - \theta_{1} \nonumber }\pause 
\end{eqnarray}

\invisible<1-4>{The pmf for $\boldsymbol{X}_{i}$ is, }\pause 

\begin{eqnarray}
\invisible<1-5>{p(\boldsymbol{x}_{i}| \boldsymbol{\theta}) & = & \prod_{j=1}^{3} \theta_{j}^{x_{ij}} \nonumber \\}\pause 
\invisible<1-6>{\boldsymbol{X}_{i} & \sim & \text{Multinomial}(1, \boldsymbol{\theta}) \nonumber \\}\pause 
\invisible<1-7>{\boldsymbol{X}_{i} & \sim & \text{Categorical}(\boldsymbol{\theta}) \nonumber }
\end{eqnarray}





\end{frame}

\begin{frame}
\frametitle{Unigram Model of Language}


\begin{eqnarray}
p(\boldsymbol{x}_{i}| \boldsymbol{\theta}) & = & \prod_{j=1}^{3} \theta_{j}^{x_{ij}} \nonumber \\
\boldsymbol{X}_{i} & \sim & \text{Multinomial}(1, \boldsymbol{\theta}) \nonumber \pause \\
\invisible<1>{E[x_{ij}] & = & \theta_{j}  \nonumber \\} \pause
\invisible<1-2>{\text{Var}(X_{ij}) & = & \theta_{j} (1 - \theta_{j} ) \nonumber \\} \pause 
\invisible<1-3>{\text{Cov}(x_{ij}, x_{ik}) & = &  - \theta_{j} \theta_{k} \nonumber } 
\end{eqnarray}



\end{frame}

\begin{frame}
\frametitle{Unigram Model of Language}

Suppose we make $N$ independent draws:
\begin{eqnarray}
\boldsymbol{X}_{i} & \sim & \text{Multinomial}(1, \boldsymbol{\theta})  \nonumber 
\end{eqnarray}

Then:
\begin{eqnarray}
\boldsymbol{X} & = &  \sum_{i=1}^{N}\boldsymbol{X}_{i} \nonumber \\
& = &   (\sum_{i=1}^{N} X_{i1}, \sum_{i=1}^{N}X_{i2}, \sum_{i=1}^{N} X_{i3} ) \nonumber \\
\boldsymbol{X} & \sim & \text{Multinomial}(N, \boldsymbol{\theta}) \nonumber \\
p(\boldsymbol{x}|\boldsymbol{\theta} ) & \propto & \prod_{j=1}^{3} \theta_{j}^{x_{j}} \nonumber 
\end{eqnarray}

\end{frame}

\begin{frame}
\frametitle{Unigram Model of Language}

Obtaining maximum-likelihood estimates:

\begin{eqnarray}
\mathcal{L}(\boldsymbol{\theta}| \boldsymbol{x} ) & \propto & \prod_{j=1}^{3} \theta_{j}^{x_{j} } \nonumber \\
\log \mathcal{L}(\boldsymbol{\theta}| \boldsymbol{x} ) & =  & \sum_{j=1}^{3} x_{j} \log \theta_{j} + c  \nonumber 
\end{eqnarray}

Include constraint that $\sum_{j=1}^{3} \theta_{j} = 1$ 

\begin{eqnarray}
\log \mathcal{L}(\boldsymbol{\theta}| \boldsymbol{x} ) & =  & \sum_{j=1}^{3} x_{j} \log \theta_{j} + \lambda(\sum_{j=1}^{3} \theta_{j} - 1)   + c\nonumber 
\end{eqnarray}



\end{frame}


\begin{frame}
\frametitle{Unigram Model of Language}

\begin{eqnarray}
\log \mathcal{L}(\boldsymbol{\theta}| \boldsymbol{x} ) & =  & \sum_{j=1}^{3} x_{j} \log \theta_{j} + \lambda(\sum_{j=1}^{3} \theta_{j} - 1)   + c \nonumber \\
\frac{\partial \log \mathcal{L}(\boldsymbol{\theta}| \boldsymbol{x} )}{\partial \theta_{1}} & = & \frac{x_{1}}{\theta_{1}} + \lambda \nonumber \\
\frac{\partial \log \mathcal{L}(\boldsymbol{\theta}| \boldsymbol{x} )}{\partial \theta_{2}} & = & \frac{x_{2}}{\theta_{2}} + \lambda \nonumber \\
\frac{\partial \log \mathcal{L}(\boldsymbol{\theta}| \boldsymbol{x} )}{\partial \theta_{3}} & = & \frac{x_{3}}{\theta_{3}} + \lambda \nonumber \\
\frac{\partial \log \mathcal{L}(\boldsymbol{\theta}| \boldsymbol{x} )}{\partial \lambda} & = & \sum_{j=1}^{3}\theta_{j} - 1 \nonumber 
\end{eqnarray}
\end{frame}


\begin{frame}
\frametitle{Unigram Model of Language}

\Large 

\only<1>{
\begin{eqnarray}
0 & = & \frac{x_{1}}{\theta_{1}^{*}} + \lambda^{*} \nonumber \\
0 & = & \frac{x_{2}}{\theta_{2}^{*}} + \lambda^{*} \nonumber \\
0 & = & \frac{x_{3}}{\theta_{3}^{*}} + \lambda^{*} \nonumber \\
0 & = & \sum_{j=1}^{3}\theta_{j}^{*} - 1 \nonumber 
\end{eqnarray}}

\only<2->{
\begin{eqnarray}
\theta_{1}^{*} & = & \frac{x_{1}}{x_{1} + x_{2} + x_{3} } \nonumber \\
\theta_{2}^{*} & = & \frac{x_{2}}{x_{1} + x_{2} + x_{3} } \nonumber \\
\theta_{3}^{*} & = & \frac{x_{3}}{x_{1} + x_{2} + x_{3} } \nonumber 
\end{eqnarray}	
\invisible<1-2>{Maximum likelihood estimates$\leadsto$ Rates words are used}
}
\pause \pause 

\end{frame}


\begin{frame}
\frametitle{Unigram Model of Language}


\begin{eqnarray}
p(\boldsymbol{x}|\boldsymbol{\theta} ) & \propto & \prod_{j=1}^{3} \theta_{j}^{x_{j}} \nonumber 
\end{eqnarray}
\pause 

\invisible<1>{$\alert{\boldsymbol{\theta}}$: encodes information about word rates$\leadsto$ our summary of the document/speaker} \pause 
\begin{itemize}
\invisible<1-2>{\item[-] $\sum_{j=1}^{3} \theta_{j} = 1$} \pause 
\invisible<1-3>{\item[-] $\theta_{j}\geq 0$ } \pause 
\end{itemize}

\invisible<1-4>{$\boldsymbol{\theta} \in \Delta^{2}$ (2-dimensional simplex )} 


\end{frame}


\begin{frame}


\begin{center}
\only<1>{\scalebox{0.55}{\includegraphics{Simplex3d.pdf}}}
\end{center}


\end{frame}


\begin{frame}

\begin{tikzpicture}
\node (empty) at (-8, 5) [] {} ; 

\node (figure) at (-7, 7) [] {\scalebox{0.55}{\includegraphics{TernaryPlot1.pdf}}}; 

\invisible<1, 3->{\node (corner) at (-9.75, 4.3) [] {}; 
\node (corner2) at (-11, 5) [] {}; 

\draw[->, line width = 2pt]  (corner2) to [out = 180, in = 90] (corner) ; 
\node(corner1) at (-11, 6.25) [] {\scalebox{0.25}{\includegraphics{Corner2.pdf}}}; 
}

\invisible<1-2, 4->{\node (corner3) at (-6.65, 9.7) [] {}; 
\node (corner4) at (-4.75, 10.2) [] {}; 

\draw[->, line width = 2pt]  (corner4) to [out = 180, in = 90] (corner3) ; 
\node(corner_pic) at (-4, 10.25) [] {\scalebox{0.25}{\includegraphics{Corner1.pdf}}}; 
}

\invisible<1-3, 5->{\node (corner5) at (-3.7, 4.3) [] {}; 
\node (corner6) at (-2.5, 5) [] {}; 

\draw[->, line width = 2pt]  (corner6) to [out = 180, in = 90] (corner5) ; 
\node(corner_pic2) at (-2.5, 6.25) [] {\scalebox{0.25}{\includegraphics{Corner3.pdf}}}; 
}


\invisible<1-4>{\node (corner7) at (-6.7, 6.3) [] {}; 
\node (corner8) at (-10.5, 10) [] {}; 

\draw[->, line width = 2pt]  (corner8) to [out = 270, in = 90] (corner7) ; 
\node(corner_center) at (-10.5, 10.25) [] {\scalebox{0.25}{\includegraphics{Corner4.pdf}}}; 
}

\end{tikzpicture}
\pause \pause \pause \pause 

\end{frame}

\begin{frame}
\frametitle{Senate Press Releases 2005}
\begin{center}
\scalebox{0.5}{\includegraphics{OverallMLE.pdf}}
\end{center}
\end{frame}


\begin{frame}
\frametitle{Unigram Model of Language}

\Large 

Suppose we have several speakers (authors/clusters/topics/categories/ ...)\\

Speaker $i$ produces document $\boldsymbol{X}_{i}$, 
\begin{eqnarray}
\boldsymbol{X}_{i} & \sim & \text{Multinomial}(N_{i}, \alert{\boldsymbol{\theta}_{i}} ) \nonumber 
\end{eqnarray}

where $\boldsymbol{\theta}_{i} \leadsto$ Speaker specific word rates\\

Build hierarchical model: 
\begin{eqnarray}
\boldsymbol{\theta}_{i} & \sim & \text{Distribution on Simplex} \nonumber 
\end{eqnarray}




\end{frame}


\begin{frame}
\frametitle{Hierarchical Models as a Modeling Paradigm}


Why Build a Hierarchical Model?

\begin{itemize}
\item[1)] Borrow strength across documents$\leadsto$ Improved and granular inferences
\item[2)] Shrink estimates$\leadsto$ regularization
\item[3)] Incorporate further covariate information
\begin{itemize}
\item[i)] Author
\item[ii)] Time
\item[iii)] ...
\end{itemize}
\item[3)] Learn additional structure 
\begin{itemize}
\item[i)] Hierarchies of word rates
\item[ii)] Clusters of similar word rates
\item[iii)] Low dimensional approximations of word rates
\end{itemize}
\item[4)] Encodes complicated dependencies between documents/speakers

\end{itemize}



\end{frame}


\begin{frame}
\frametitle{Dirichlet-Multinomial Unigram Language Model}

For $N$ observations we observe a $3$-element long count vector 
\begin{eqnarray}
\boldsymbol{x}_{i} & = &  (x_{i1}, x_{i2}, x_{i3}) \nonumber 
\end{eqnarray}


Where $N_{i} = \sum_{j=1}^{3} x_{ij}$ .\\

Suppose 


\begin{eqnarray}
\boldsymbol{\theta}_{i} & \sim & \text{Dirichlet}(\boldsymbol{\alpha}) \nonumber \\
\boldsymbol{x}_{i} | \boldsymbol{\theta}_{i} & \sim & \text{Multinomial}(N_{i}, \boldsymbol{\theta}_{i} ) \nonumber 
\end{eqnarray}

\begin{itemize}
\item[-] Dirichlet distribution$
\leadsto$ assumption about \alert{population} of word rates
\item[-] $\boldsymbol{\alpha}= (\alpha_{1}, \alpha_{2}, \alpha_{3})$ describes population use of words and variation
\item[-] \alert{Just one distribution simplex}
\end{itemize}



\end{frame}



\begin{frame}

\only<1>{\scalebox{0.5}{\includegraphics{Dirichlet1.pdf}}}
\only<2>{\scalebox{0.5}{\includegraphics{Dirichlet2.pdf}}}
\only<3>{\scalebox{0.5}{\includegraphics{Dirichlet3.pdf}}}
\only<4>{\scalebox{0.5}{\includegraphics{Dirichlet4.pdf}}}
\only<5>{\scalebox{0.5}{\includegraphics{Dirichlet5.pdf}}}
\only<6>{\scalebox{0.5}{\includegraphics{Dirichlet6.pdf}}}
\only<7>{\scalebox{0.5}{\includegraphics{Dirichlet7.pdf}}}
\only<8>{\scalebox{0.5}{\includegraphics{Dirichlet8.pdf}}}
\only<9>{\scalebox{0.5}{\includegraphics{Dirichlet9.pdf}}}
\only<10>{\scalebox{0.5}{\includegraphics{Dirichlet10.pdf}}}

\end{frame}



\begin{frame}
\frametitle{Dirichlet Distribution}

Suppose 
\begin{eqnarray}
\boldsymbol{\theta}_{i} & \sim & \text{Dirichlet}(\boldsymbol{\alpha}) \nonumber 
\end{eqnarray}

Then, 
\begin{eqnarray}
p(\boldsymbol{\theta}|\boldsymbol{\alpha}) & = & \frac{\Gamma(\sum_{j=1}^{3} \alpha_{j})}{\prod_{j=1}^{3} \Gamma(\alpha_{j})} \prod_{j=1}^{3}  \theta_{ij}^{\alpha_{j} - 1}  \nonumber 
\end{eqnarray}

\begin{itemize}
\item[-] If $\boldsymbol{\alpha} = (1, 1, 1)$ \alert{Uniform distribution}
\end{itemize}



\end{frame}


\begin{frame}
\frametitle{Dirichlet Distribution}

\begin{itemize}
\item[-] Important Facts
\begin{eqnarray}
E[\boldsymbol{\theta}_{i}] & = &  \left(\frac{\alpha_{1} }{\sum_{j=1}^{3} \alpha_{j} }, \frac{\alpha_{2} }{\sum_{j=1}^{3} \alpha_{j} }, \frac{\alpha_{3} }{\sum_{j=1}^{3} \alpha_{j} }\right) \nonumber \\
\text{var}(\theta_{ij} ) & = & \frac{\alpha_{i}\left(\sum_{j=1}^{3} \alpha_{j} - \alpha_{i} \right)}{\left(\sum_{j=1}^{3} \alpha_{j}\right)^{2} \left(\sum_{j=1}^{3} \alpha_{j} + 1 \right)} \nonumber \\
\text{cov}(\theta_{ik}, \theta_{ij} ) & = & \frac{-\alpha_{k} \alpha_{j} }{ \left(\sum_{j=1}^{3} \alpha_{j}\right)^{2} \left(\sum_{j=1}^{3} \alpha_{j} + 1 \right)  } \nonumber \\
\text{Mode}(\theta_{j}) & = & \frac{\alpha_{j}  - 1}{\sum_{k=1}^{3} \alpha_{k} - 3} \nonumber 
\end{eqnarray}
\end{itemize}

\end{frame}

\begin{frame}
\frametitle{Dirichlet-Multinomial Unigram Model of Language}

\begin{eqnarray}
\boldsymbol{\theta}_{i} & \sim & \text{Dirichlet}(\boldsymbol{\alpha}) \nonumber \\
\boldsymbol{x}_{i} | \boldsymbol{\theta}_{i} & \sim & \text{Multinomial}(N_{i}, \boldsymbol{\theta}_{i}) \nonumber 
\end{eqnarray}
\pause 

\begin{eqnarray}
\invisible<1>{p(\boldsymbol{\theta}_{i}| \boldsymbol{\alpha}, \boldsymbol{x}_{i} ) &  \propto & p(\boldsymbol{\theta}_{i} | \boldsymbol{\alpha}) \ p(\boldsymbol{x}_{i}| \boldsymbol{\theta}_{i}) \nonumber \\} \pause 
\invisible<1-2>{& \propto &  \frac{\Gamma(\sum_{j=1}^{3} \alpha_{j}) }{\prod_{j=1}^{3} \Gamma(\alpha_{j})} \prod_{j=1}^{3} \theta_{j}^{\alpha_{j} - 1 }  \prod_{j=1}^{3} \theta_{ij}^{x_{ij}}  \nonumber \\} \pause 
\invisible<1-3>{& \propto &  \frac{\Gamma(\sum_{j=1}^{3} \alpha_{j}) }{\prod_{j=1}^{3} \Gamma(\alpha_{j})} \underbrace{\prod_{j=1}^{3} \theta_{j}^{\alpha_{j} + x_{ij}  - 1 }}_{\text{Dirichlet Kernel}} \nonumber }
\end{eqnarray}

\end{frame}



\begin{frame}
\frametitle{Dirichlet-Multinomial Unigram Model of Language}


\begin{eqnarray}
\boldsymbol{\theta}_{i}| \boldsymbol{\alpha}, \boldsymbol{x}_{i} & \sim & \text{Dirichlet}(\boldsymbol{\alpha} + \boldsymbol{x} ) \nonumber \\
\text{E}[\theta_{ij}| \boldsymbol{\alpha}, \boldsymbol{x}_{i}] & = & \frac{\alpha_{j} + x_{ij} }{\sum_{j=1}^{3} (x_{ij} + \alpha_{j} ) } \nonumber 
\end{eqnarray}

\begin{itemize}
\item[-] $\alpha_{j} \leadsto$ ``pseudo" data that smooth the estimates toward $\frac{\alpha_{j} }{\alpha_{1} + \alpha_{2} + \alpha_{3} }$
\item[-] as $N_{i} \rightarrow \infty$ data ($\boldsymbol{x}_{i}$) \alert{overwhelm} $\boldsymbol{\alpha}$ 
\end{itemize}




\end{frame}

\begin{frame}
\frametitle{Senate Press Releases}
\begin{center}
\only<1>{\scalebox{0.5}{\includegraphics{ObamaMcCainMLE.pdf}}}
\only<2>{\scalebox{0.5}{\includegraphics{ObamaClintonMLE.pdf}}}
\end{center}

\end{frame}



\begin{frame}
\frametitle{Dirichlet-Multinomial Unigram Model of Language}

Data generation process suggests new probability mass function for $\boldsymbol{x}_{i} \leadsto $ marginalize over $\boldsymbol{\theta}$ 

\begin{eqnarray}
\boldsymbol{\theta}_{i}  & \sim & \text{Dirichlet}(\boldsymbol{\alpha}) \nonumber \\
\boldsymbol{x}_{i}| \boldsymbol{\theta}_{i} & \sim  & \text{Multinomial}(N_{i}, \boldsymbol{\theta}_{i}) \nonumber
\end{eqnarray}

\pause 
\invisible<1>{This implies:} \pause 
\begin{eqnarray}
\invisible<1-2>{p(\boldsymbol{x}_{i} | \boldsymbol{\alpha}) & =  & \int_{\Delta^{2}} p(\boldsymbol{x}_{i}, \boldsymbol{\theta}_{i}|\boldsymbol{\alpha})d\boldsymbol{\theta} \nonumber \\}\pause 
 \invisible<1-3>{& =  & \int_{\Delta^{2}} p(\boldsymbol{x}_{i} | \boldsymbol{\theta}_{i} ) p(\boldsymbol{\theta}_{i}| \boldsymbol{\alpha} ) d\boldsymbol{\theta} \nonumber \\} \pause 
 \invisible<1-4>{& = & {{N_{i}}\choose{n_{1}!n_{2}!n_{3}!}} \frac{\Gamma(\sum_{j=1}^{3} \alpha_{j} )}{\prod_{j=1}^{3} \Gamma(\alpha_{j}) }  \int_{\Delta^{2}} \prod_{j=1}^{3} \theta_{ij}^{x_{ij}} \theta_{ij}^{\alpha_{j} - 1} d\boldsymbol{\theta}  \nonumber }
\end{eqnarray}


\end{frame}



\begin{frame}

\begin{eqnarray}
\invisible<1>{p(\boldsymbol{x}_{i} | \boldsymbol{\alpha}) & =  & {{N_{i}}\choose{n_{1}!n_{2}!n_{3}!}} \frac{\Gamma(\sum_{j=1}^{3} \alpha_{j} )}{\prod_{j=1}^{3} \Gamma(\alpha_{j}) }  \int_{\Delta^{2}} \underbrace{\prod_{j=1}^{3} \theta_{ij}^{x_{ij}} \theta_{ij}^{\alpha_{j} - 1} d\boldsymbol{\theta} }_{\text{Dirichlet Kernel}} \nonumber \\}
\invisible<1-2>{& = & {{N_{i}}\choose{n_{1}!n_{2}!n_{3}!}} \frac{\Gamma(\sum_{j=1}^{3} \alpha_{j} )}{\prod_{j=1}^{3} \Gamma(\alpha_{j}) } \frac{\prod_{j=1}^{3} \Gamma(x_{ij} + \alpha_{j} )}{\Gamma(\sum_{j=1}^{3} (x_{ij} + \alpha_{j})   )    }\nonumber \\} 
\invisible<1-3>{& = & {{N_{i}}\choose{n_{1}!n_{2}!n_{3}!}} \frac{\Gamma(\sum_{j=1}^{3} \alpha_{j} )  }{ \Gamma(\sum_{j=1}^{3} (x_{ij} + \alpha_{j} )  )  } \prod_{j=1}^{3} \frac{ \Gamma(x_{ij} + \alpha_{j} )   }{ \Gamma(\alpha_{j} )  }  \nonumber }
\end{eqnarray}

\pause \pause \pause \pause 

\invisible<1-4>{Has some intuitive properties  } \pause 

\begin{eqnarray}
\invisible<1-5>{\text{E}[X_{ij}] & = & N \frac{\alpha_{j} }{\sum_{k=1}^{3} \alpha_{k} } \nonumber } 
\end{eqnarray}


\end{frame}

\begin{frame}
\frametitle{Dirichlet-Multinomial Unigram Model}

We can also generate a predictive distribution$\leadsto$ probability next word is $j$ 
\begin{eqnarray}
P(\tilde{x} = 1| \boldsymbol{x}_{i}, \boldsymbol{\alpha} ) & = & \int_{\Delta^{2}} p(\tilde{x} = 1 | \boldsymbol{\theta}) p(\boldsymbol{\theta}| \boldsymbol{\alpha}, \boldsymbol{x}_{i} ) d\boldsymbol{\theta} \nonumber \\
& = & \int_{\Delta^{2}} \theta_{j} \text{Dir}(\boldsymbol{\theta}| \boldsymbol{x}_{i} + \boldsymbol{\alpha})d\boldsymbol{\theta}  \nonumber \\
& = & \frac{x_{ij} + \alpha_{j} }{ \sum_{j=1}^{3}(x_{ij} + \alpha_{j})  } \nonumber 
\end{eqnarray}


\end{frame}


\begin{frame}
\frametitle{Dirichlet-Multinomial Unigram Model}

Where does $\boldsymbol{\alpha}$ come from?\\
Extend the model$\leadsto$ \alert{infer} it

\begin{eqnarray}
\alpha_{j} & \sim & \text{Gamma}(0.25, 1) \nonumber \\
\boldsymbol{\theta}_{i}|\boldsymbol{\alpha} & \sim & \text{Dirichlet}(\boldsymbol{\theta}_{i}) \nonumber \\
\boldsymbol{x}_{i}| \boldsymbol{\theta}_{i} & \sim & \text{Multinomial}(N_{i}, \boldsymbol{\theta}_{i} ) \nonumber 
\end{eqnarray}



\end{frame}

\begin{frame}
\frametitle{Dirichlet-Multinomial Unigram Model}

Which yields
\begin{footnotesize}
\begin{eqnarray}
p(\boldsymbol{\theta}, \boldsymbol{\alpha} | \boldsymbol{X}) & \propto &  p(\boldsymbol{\alpha}) \prod_{i=1}^{N} p(\boldsymbol{\theta}_{i} | \boldsymbol{\alpha}) p(\boldsymbol{x}_{i} | \boldsymbol{\alpha})  \nonumber \\
p(\boldsymbol{\alpha}|\boldsymbol{X}) & = & \int_{\Delta^{2}} p(\boldsymbol{\theta}, \boldsymbol{\alpha} | \boldsymbol{X})  d\boldsymbol{\theta}  \nonumber \\
& \propto & 
p(\boldsymbol{\alpha}) \prod_{i=1}^{N} \int_{\Delta^{2}} p(\boldsymbol{\theta}_{i} | \boldsymbol{\alpha}) p(\boldsymbol{x}_{i} |\boldsymbol{\theta}_{i}, \boldsymbol{\alpha})d\boldsymbol{\theta} \nonumber \\
& \propto & \prod_{j=1}^{3} 4 \exp\left( - 4 \alpha_{j}  \right)  \times \prod_{i=1}^{N}\left[\frac{\Gamma(\sum_{j=1}^{3} \alpha_{j} )}{\Gamma(\sum_{j=1}^{3} (x_{ij} + \alpha_{j} ))}\times \prod_{j=1}^{3} \frac{\Gamma(x_{ij} + \alpha_{j}) }{\Gamma(\alpha_{j} )   }\right] \nonumber 
\end{eqnarray}
\end{footnotesize}




\end{frame}




\begin{frame}
\frametitle{Unigram Model of Language}

Suppose $\boldsymbol{x}_{i} = (x_{i1}, x_{i2}, \hdots, x_{iJ})$ \\
\begin{itemize}
\item[-] $x_{ij} = $ Number of times word $j$ occurs in document $i$.  
\item[-] $N_{i} = \sum_{j=1}^{J} x_{ij}$ total number of words in document $i$
\end{itemize}

Assume a generation process
\begin{eqnarray}
\boldsymbol{x}_{i} & \sim & \text{Multinomial}(N_{i}, \boldsymbol{\theta}) \nonumber \\
\boldsymbol{\theta} & = & (\theta_{1}, \theta_{2}, \hdots, \theta_{J}) \nonumber \\
p(\boldsymbol{x}_{i}| \boldsymbol{\theta}) & \propto & \prod_{j=1}^{J} \theta_{j}^{x_{ij}} \nonumber 
\end{eqnarray}





\end{frame}




\begin{frame}
\frametitle{Alternative Priors on the Simplex}

Dirichlet distribution
\begin{itemize}
\item[-] Imposes specific form on variance
\item[-] Imposes negative correlation between all components.
\item[-] We might expect some word rates to positively covary. \end{itemize}

Alternative$\leadsto$ \alert{Logistic-Normal} distribution


\end{frame}

\begin{frame}
\frametitle{Logistic-Normal Distribution}

Suppose $\boldsymbol{x}_{i} = (x_{i1}, x_{i2}, x_{i3})$.  

Define: 

\begin{eqnarray}
\boldsymbol{y}_{i} & = & \left(\log\left(\frac{x_{i1}}{x_{i3}}\right), \log\left(\frac{x_{i2}}{x_{i3}} \right)  \right ) \nonumber \\
\boldsymbol{y}_{i} & \sim & \text{Normal}(\boldsymbol{\mu}, \boldsymbol{\Sigma} )  \nonumber \\
\boldsymbol{\mu} & =  & (\mu_{1}, \mu_{2} ) \nonumber \\
\boldsymbol{\Sigma} &  = &  \begin{pmatrix} \sigma_{1}^{2} & \text{cov}(y_{i1}, y_{i2} ) \\
 \text{cov}(y_{i1}, y_{i2} ) & \sigma_{2}^{2}  \end{pmatrix}\nonumber 
\end{eqnarray} 


\end{frame}


\begin{frame}
\frametitle{Logistic-Normal Distribution}

To get back original data apply:

\begin{eqnarray}
x_{i1} & = & \left(\frac{\exp(y_{i1})}{\exp(y_{i1}) + \exp(y_{i2}) + 1}  \right) \nonumber \\
x_{i2} & = & \left(\frac{\exp(y_{i2})}{\exp(y_{i1}) + \exp(y_{i2}) + 1}  \right) \nonumber \\
x_{i3} & = & \left(\frac{1}{\exp(y_{i1}) + \exp(y_{i2}) + 1}  \right) \nonumber \\
\boldsymbol{x}_{i} & = & g(\boldsymbol{y}_{i} ) \nonumber 
\end{eqnarray}



\end{frame}


\begin{frame}
\frametitle{Logistic-Normal Distribution}

An alternative hierarchical model:
\begin{eqnarray}
\boldsymbol{\eta}_{i} & \sim & \text{MVN}(\boldsymbol{\mu}, \boldsymbol{\Sigma}) \nonumber \\
\boldsymbol{\theta}_{i} & = & g(\boldsymbol{\eta}_{i} ) \nonumber \\
\boldsymbol{x}_{i} & \sim & \text{Multinomial}(N_{i}, \boldsymbol{\theta}_{i}) \nonumber 
\end{eqnarray}

Widely used:
\begin{itemize}
\item[-] Correlated models
\item[-] Natural way to encode \alert{regressions} in prior
\end{itemize}


\end{frame}

\begin{frame}

Next week: clustering!

\end{frame}


\end{document}
